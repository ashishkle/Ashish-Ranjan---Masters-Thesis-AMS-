

%% This file contains only a table.
%% this file is included into Chapter8

\begin{table}[ht]
    \setlength{\arrayrulewidth}{0.1mm}
    \setlength{\tabcolsep}{4pt}
    \renewcommand{\arraystretch}{1.0}
    
    \centering{}
    
    \caption{GSS Contribution in BIS Artefact adapted from \citep{bobbert2017exploring}}
    \label{table:GSS}
    
    \begin{tabularx}{0.96\linewidth}{|>{\columncolor[HTML]{ECB4E8}} p{1.5cm}|p{11.55cm}|} 
    
%    |a|>{\columncolor[HTML]{FFFFFF}}C|C|C|
    \arrayrulecolor[HTML]{06000A}
    
        %% Table Body
        \hline
       
         \rowcolor[HTML]{BFCEED}     
         \textbf{Type of research within DSR} 
         & 
         \textbf{Contribution to designing and engineering a Business Information Security artefact}
         \\
        \hline
        Literature research 
        &
        Explicating and defining the problem in a systematic, structured way. Objectivity removes the
        element of Fear Uncertainty and Doubt (FUD). Unbiased, structured point of departure for the
        design cycle. Requires a certain level of expertise in the topic.
         \\
        \hline
        Delphi research
        &
        Anonymous inventory and selection of views and standpoints (preferably based upon
literature data). Rigorous examination process for scrutinizing the problem via, for example,
expert opinions. Collecting global views on criteria requirements with the use of technology.
Knowledge sharing. Enables double loop learning via multiple iterations. Automated. No
geographical limitations. Limited in group interaction and discussion.
         \\
        \hline
        Group Support System research 
        & 
        Enables to create, share and capture knowledge as well as design items.
        Stimulates design thinking and stakeholder collaboration due to the “group element”. 
        Ability to collect, assess and select product requirements in a very short time frame. 
        Supports the regulative process of testing and validating requirements. 
        Processing large data sets.
        Double Loop learning. 
        Bridging knowing-doing gaps. 
        Stimulating group dialogues (i.e. among Boards of Directors and Management teams).
        Makes it possible to establish group consensus. 
        Supports the decision making process. 
        Threat of the “law of the decibel”. 
        Requires professional group moderation skills.  	
        \\
       \hline
    \end{tabularx}

\end{table}








