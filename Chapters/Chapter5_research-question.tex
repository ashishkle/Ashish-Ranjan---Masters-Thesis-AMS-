% Chapter 5

\chapter{Research question} % Main chapter title

\label{Chapter5_research-question} % For referencing the chapter elsewhere, use \ref{Chapter5} 

%----------------------------------------------------------------------------------------

The scenario here is about the retrieving cyber threat information to transform into cyber intelligence over various different sources. Would it be beneficial for the stakeholders if,  we could design an prototype artefact 1) to overcome the volumetric problem of sources, 2) to filter the cyber newsfeed into a specific context, and 3) to tag relatable data stream for automatic risk determination?

As an Example: For any vital sector\citep{luiijf2003critical}[table 2] like water\footnote{\url{https://en.wikipedia.org/wiki/Water_industry}} industry, may only like to receive water industry-specific cyber newsfeeds. A water industry may not be interested in hospital or oil or any other industry-specific cyber newsfeeds. 
\bigbreak
\textbf{How can we create a prototype, for assessing the quality of the cyber news data sources, for filtering contextual cyber newsfeed, for tagging relatable cybernews feed stream, cybernews Correlation, for performing Relevance tagging and Automated Risk Determination, which can be validated by the users of the artefact to make it relevant for a specific organization?}

\section{Sub-research question}
To answer the aforementioned research question, we need to disseminate the research question into logically  sub-research questions to focus on distinct part of prototype to make it beneficial for usages. With this problem field in mind we define three related sub-research questions:

\begin{enumerate}
    \item What are elements for a cybernews feed assessment method and what are the parameters for automation?
    \begin{enumerate}
        \item What are the core concepts of Cybernews feeds and how do they work according to the literature?
        
        \item How does the vendor field of automated cybernewsfeed suppliers of artefacts look like according to the literature?
    \end{enumerate}
    
    \item How to make a prototype for assessing the quality of the data sources?

    \item How to make a prototype to filter contextual cyber Intelligence, perform automatic relevance tagging and perform analysis for automated risk determination and advisory?
    
    \item How to get the prototype validated for \emph{both 1) the design and 2) the output of this artefact} by the end users of specific organizations?
\end{enumerate}


\section{Research Deliverable:}
An Implemented prototype for converting cyber newsfeed or cyber information into relevant and contextual cyber intelligence based on parameterized context and tags with results duly validated by end users.
\begin{enumerate}
    \item For Sub-research Question 3: A prototype which will have modules or sub-modules for collection, processing, analysis, publication and collection of user feedback.
    \item For Sub-research Question 2: A prototype to access the quality of data sources.
    \item For Sub-research Question 4: A demo of the simulated prototype and validation report from group of experts.
\end{enumerate}

This prototype was developed as software in coordination with ON2IT research department. An output of this simulation will be verified for the relevancy and actionability in the context of stakeholder’s interest.


