% Appendix Template

\chapter{JSON newsfeed format to Table} % Main appendix title
\section{ON2IT Vendor Context}
\label{json} % Change X to a consecutive letter; for referencing this appendix elsewhere, use \ref{json}
\label{AppendixB} % For referencing this appendix elsewhere, use \ref{AppendixA}

\begin{table}[htbp!]
   \setlength{\arrayrulewidth}{0.1mm}
    \setlength{\tabcolsep}{5pt}
    \renewcommand{\arraystretch}{1.0}
\label{tab:json}
  %  \centering
   \resizebox{1.1\textwidth}{!}{ \begin{tabular}{|p{1.3cm}|p{1.5cm}|p{9.7cm}|p{4cm}|p{1.7cm}| p{4.3cm}|}
    \hline
        TAGS &  on2itcontext & newsfeed & title & pub\_date & link \\ \hline
        \#microsoft & VENDOR & Newly-elected politicians in Munich "have decided its administration needs to use open-source software, instead of proprietary products like Microsoft Office," reports ZDNet:

Munich began the move away from proprietary software at the end of 2006... By 2013, 80\% of desktops in the city's administration were meant to be running LiMux software. In reality, the council continued to run the two systems — Microsoft and LiMux — side by side for several years to deal with compatibility issues. As the result of a change in the city's government, a controversial decision was made in 2017 to leave LiMux and move back to Microsoft by 2020. At the time, critics of the decision blamed the mayor and deputy mayor and cast a suspicious eye on the US software giant's decision to move its headquarters to Munich. In interviews, a former Munich mayor, under whose administration the LiMux program began, has been candid about the efforts Microsoft went to to retain their contract with the city. 

The migration back to Microsoft and to other proprietary software makers like Oracle and SAP, costing an estimated 86 million euro, is still in progress today. 



 & Munich Says It's Now Shifting Back From Microsoft to Open Source Software -- Again & 2020-05-24 & http://rss.slashdot.org/
 \~r/Slashdot/slashdotYou
 rRightsOn
 line/\~3/KSXgvcOs5oM/
 munich-says-its-now-shifting-back-from-microsoft-to-open-source-software----again\\ \hline
        \#microsoft &   VENDOR & <p>Allegations that the North Dakota COVID-19 contact tracing app, Care19, shares data with location platform Foursquare are being refuted, MediaPost reports. Gov. Doug Burgum, R-N.D., said the app does not require or utilize names, addresses, emails, phone numbers or other personal information, and location data is held securely.The data is not being shared or sold for commercial purposes,he added.<br><a href="https://www.mediapost.com/publications/article
        /351783/north-dakota-microsoft-developer-for-covid-tracing.html" target="\_blank" rel="noopener noreferrer"><strong>Full Story</strong></a></p> & Data-sharing allegations against COVID-19 contact tracing app refuted & 2020-05-26 & https://iapp.org/news/a
        /data-sharing-allegations-against-covid-19-contact-tracing-app-refuted \\ \hline
    \end{tabular}
    }
\end{table}


\section{ON2IT Non-Vendor Context}
\label{json-n} % Change X to a consecutive letter; for referencing this appendix elsewhere, use \ref{json}
\begin{table}[htbp!]
   \setlength{\arrayrulewidth}{0.1mm}
    \setlength{\tabcolsep}{5pt}
    \renewcommand{\arraystretch}{1.0}
\label{tab:json-n}
  %  \centering
   \resizebox{1.1\textwidth}{!}{ \begin{tabular}{|p{1.3cm}|p{1.5cm}|p{9.7cm}|p{4cm}|p{1.7cm}| p{4.3cm}|}
    \hline
        TAGS &  on2itcontext & newsfeed & title & pub\_date & link \\ \hline
        \# 
& Non-VENDOR 
& advice for users of whatsapp following today's vulnerability announcement
 & NCSC advice following WhatsApp vulnerability 
 & 2020-05-24 
 & https://www.ncsc.gov.uk/
 guidance/whatsapp-vulnerability\\ 
 \hline
        \#
&   Non-VENDOR 
& 
        
find out about how netscout solutions seamlessly integrate with microsoft azure vtap for greater insight into application and network performance.<div class="enclosure"><p class="enclosure-content"><img src="https://www.netscout.com/sites/default/files/shutterstock\_420389020.jpg" alt="" /></p></div>

& Optimizing Application Performance and User Experience with NETSCOUT for Azure
& 2020-05-26 
& https://www.netscout.
com/blog/azure-vtap 

\\ \hline
    \end{tabular}
    }
\end{table}